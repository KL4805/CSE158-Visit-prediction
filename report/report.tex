\documentclass{article}
\usepackage{enumerate}
\usepackage{geometry}
\usepackage{amsmath}
\usepackage{fancyhdr}
\usepackage{graphicx}
\geometry{a4paper,scale=0.7}
\title{\bf CSE158 Assignment 1}
\author{\bf Yilun Jin, A15361514}
\setlength{\baselineskip}{24pt}

\pagestyle{fancy}
\lhead{}
\chead{}
\rhead{\bfseries Report for Google Local Predictions}
\lfoot{Yilun Jin}
\cfoot{Part 1: Visit Prediction}
\rfoot{\thepage}
\renewcommand{\headrulewidth}{0.4pt}
\renewcommand{\footrulewidth}{0.4pt}

 

\begin{document}
%titlepage
\thispagestyle{empty}
\begin{center}
\begin{minipage}{0.75\linewidth}
    \centering


%Thesis title
	\vspace{4cm}
    {{\LARGE Report for Google local visit prediction and category prediction\par}}
    \vspace{4cm}
%Author's name
    {\Large Yilun Jin, A15361514\par}
    \vspace{4cm}
%Degree
    {\Large A report submitted for course CSE 158, UC San Diego\par}
    \vspace{4cm}
%Date
    {\Large November 2017}
\end{minipage}
\end{center}
\clearpage


\section{Visit Prediction}
\subsection{Method Overview}
The approach I took was basically trying to predict according to previous user activities and similarities between businesses, via a linear classification model. 

\subsection{Feature Vector}
The feature vector designed is a 21-dimensional vector comprising of the following data. 
\begin{enumerate}
\item The business's average rating. The higher it is, the more likely it will attract potential visitors. 
\item The business's popularity, i.e. total times it was visited. 
\item The user's activity level, i.e. total times the user visited businesses and left a review. 
\end{enumerate}

\end{document}

